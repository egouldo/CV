%!TEX TS-program = xelatex
%!TEX encoding = UTF-8 Unicode
% Awesome CV LaTeX Template for CV/Resume
%
% This template has been downloaded from:
% https://github.com/posquit0/Awesome-CV
%
% Author:
% Claud D. Park <posquit0.bj@gmail.com>
% http://www.posquit0.com
%
%
% Adapted to be an Rmarkdown template by Mitchell O'Hara-Wild
% 23 November 2018
%
% Template license:
% CC BY-SA 4.0 (https://creativecommons.org/licenses/by-sa/4.0/)
%
%-------------------------------------------------------------------------------
% CONFIGURATIONS
%-------------------------------------------------------------------------------
% A4 paper size by default, use 'letterpaper' for US letter
\documentclass[11pt, a4paper]{awesome-cv}

% Configure page margins with geometry
\geometry{left=1.4cm, top=.8cm, right=1.4cm, bottom=1.8cm, footskip=.5cm}

% Specify the location of the included fonts
\fontdir[fonts/]

% Color for highlights
% Awesome Colors: awesome-emerald, awesome-skyblue, awesome-red, awesome-pink, awesome-orange
%                 awesome-nephritis, awesome-concrete, awesome-darknight

\colorlet{awesome}{awesome-red}

% Colors for text
% Uncomment if you would like to specify your own color
% \definecolor{darktext}{HTML}{414141}
% \definecolor{text}{HTML}{333333}
% \definecolor{graytext}{HTML}{5D5D5D}
% \definecolor{lighttext}{HTML}{999999}

% Set false if you don't want to highlight section with awesome color
\setbool{acvSectionColorHighlight}{true}

% If you would like to change the social information separator from a pipe (|) to something else
\renewcommand{\acvHeaderSocialSep}{\quad\textbar\quad}

\def\endfirstpage{\newpage}

%-------------------------------------------------------------------------------
%	PERSONAL INFORMATION
%	Comment any of the lines below if they are not required
%-------------------------------------------------------------------------------
% Available options: circle|rectangle,edge/noedge,left/right

\name{Elliot Gould}{}

\position{PhD Candidate, Research Assistant, Consultant}
\address{School of Agriculture, Food and Ecosystem Sciences, University
of Melbourne}

\mobile{+61406680382}
\email{\href{mailto:elliot.gould@unimelb.edu.au}{\nolinkurl{elliot.gould@unimelb.edu.au}}}
\github{egouldo}

% \gitlab{gitlab-id}
% \stackoverflow{SO-id}{SO-name}
% \skype{skype-id}
% \reddit{reddit-id}

\quote{Elliot Gould is a PhD candidate at the School of Agriculture,
Food and Ecosystem Sciences, and a Quantitative Research Assistant on
the \href{replicats.research.unimelb.edu.au/}{repliCATS project} at the
School of History and Philosophical Studies, University of Melbourne.
Their PhD investigates the transparency and reproducibility of
ecological models in applied ecology and conservation decision-making.
In their role as a Quantitative Research Assistant, Elliot managed a
small team of researchers to develop a data analytics and management
platform for the repliCATS project, and contributed to research on
metascience. They have an enthusiasm for teaching and skill-sharing,
particularly with regard to building a strong community of practice in
emerging open-science methodology and computational biology within
ecology and conservation. Elliot's research seeks to use data science
techniques to advance the open-science movement by improving
transparency and reproducibility, focussing on ecology and conservation
Science. Other research interests include decision-theory, Structured
Decision Making, and plant ecology (especially grasslands of the
Victorian Volcanic Plains).}

\usepackage{booktabs}

\providecommand{\tightlist}{%
	\setlength{\itemsep}{0pt}\setlength{\parskip}{0pt}}

%------------------------------------------------------------------------------



% Pandoc CSL macros
\newlength{\cslhangindent}
\setlength{\cslhangindent}{1.5em}
\newlength{\csllabelwidth}
\setlength{\csllabelwidth}{3em}
\newenvironment{CSLReferences}[3] % #1 hanging-ident, #2 entry spacing
 {% don't indent paragraphs
  \setlength{\parindent}{0pt}
  % turn on hanging indent if param 1 is 1
  \ifodd #1 \everypar{\setlength{\hangindent}{\cslhangindent}}\ignorespaces\fi
  % set entry spacing
  \ifnum #2 > 0
  \setlength{\parskip}{#2\baselineskip}
  \fi
 }%
 {}
\usepackage{calc}
\newcommand{\CSLBlock}[1]{#1\hfill\break}
\newcommand{\CSLLeftMargin}[1]{\parbox[t]{\csllabelwidth}{#1}}
\newcommand{\CSLRightInline}[1]{\parbox[t]{\linewidth - \csllabelwidth}{#1}}
\newcommand{\CSLIndent}[1]{\hspace{\cslhangindent}#1}

\begin{document}

% Print the header with above personal informations
% Give optional argument to change alignment(C: center, L: left, R: right)
\makecvheader

% Print the footer with 3 arguments(<left>, <center>, <right>)
% Leave any of these blank if they are not needed
% 2019-02-14 Chris Umphlett - add flexibility to the document name in footer, rather than have it be static Curriculum Vitae
\makecvfooter
  {January 2024}
    {Elliot Gould~~~·~~~Curriculum Vitae}
  {\thepage}


%-------------------------------------------------------------------------------
%	CV/RESUME CONTENT
%	Each section is imported separately, open each file in turn to modify content
%------------------------------------------------------------------------------



\hypertarget{education}{%
\section{Education}\label{education}}

\begin{cventries}
    \cventry{University of Melbourne}{Doctor of Philosophy, Science}{}{November 2017 - Present}{\begin{cvitems}
\item Thesis Title: Reproducibility and Transparency of Ecological Models in Applied Ecology and Conservation Science
\end{cvitems}}
    \cventry{University of Melbourne}{Master of Science (Distinction)}{}{March 2012 - December 2015}{\begin{cvitems}
\item Research Training Degree, with 70\% original research and 30\% coursework. Course Weighted Average Mark: 82.312
\item Thesis Title: Managing Grasslands with Models: Resolving uncertainty and allocating effort among a suite of sites.
\end{cvitems}}
    \cventry{University of Melbourne}{Bachelor of Science}{}{March 2005 - November 2011}{\begin{cvitems}
\item Major in Ecology, First Class Honours Average
\end{cvitems}}
    \cventry{University of Melbourne}{Bachelor of Arts}{}{March 2005 - November 2011}{\begin{cvitems}
\item Major in Indonesian, First Class Honours Average
\end{cvitems}}
\end{cventries}

\hypertarget{employment-history}{%
\section{Employment History}\label{employment-history}}

\begin{cventries}
    \cventry{School of Historical and Philosophical Studies, School of BioSciences}{Quantitative Research Assistant - repliCATS, SCORE Program}{University of Melbourne}{February 2019 - Present}{\begin{cvitems}
\item Systematising Confidence in Open Research and Evidence (SCORE) is a Research Program initiated by the Defense Advanced Research Projects Agency  (DARPA) that aims to develop and deploy automated tools to assign 'confidence scores' to Social and Behavioural research results and claims in light of recent evidence about the 'Replication Crisis' besetting Science. The repliCATS project is one team within the SCORE project, based in Melbourne. In this role, Elliot lead a small team within the repliCATS project to build data analysis software and infrastructure to manage and deliver data products to internal teams and external partners. Research components of the role include modelling to investigate predictors of replication success.
\end{cvitems}}
    \cventry{School of BioSciences}{Demonstrator / Tutor}{University of Melbourne}{2012 - Present}{\begin{cvitems}
\item Environmental Risk Assessment: 2016, 2017, 2018, 2022, 2023
\item Guest lectures in `Biometry' (2023) and 'Critical Thinking with Data' (2021).
\item Vegetation Management and Conservation, 2018, 2019. In addition to demonstrating, I co-developed a teaching and learning module, and developed and delivered a workshop teaching the basics of data-science in R using data collected by the students.
\item Applied Ecology: 2014, 2015.
\item Ecology: 2014.
\item Biology of Cells and Organisms: 2012, 2013, 2014, 2015.
\end{cvitems}}
    \cventry{School of BioScience, School of Geography}{Research Assistant, Various Roles}{University of Melbourne}{2015 - Present}{\begin{cvitems}
\item National Environmental Science Programme, Threatened Species Recovery Hub: Conservation actions for Threatened Ecological Communities.
\end{cvitems}}
    \cventry{School of BioScience, School of Geography}{Research Assistant, Various Roles}{University of Melbourne}{2015 - 2019}{\begin{cvitems}
\item Various plant ecology and Structured Decision Making projects, involving: data analysis and visualisations, building shiny Apps, model building and testing.
\end{cvitems}}
\end{cventries}

\hypertarget{scholarships-and-awards}{%
\section{Scholarships and Awards}\label{scholarships-and-awards}}

\begin{cventries}
    \cventry{University of Melbourne, Faculty of Science, School of Ecosystem and Forest Sciences}{Science Abroad Travelling Scholarships, 2023}{}{2023}{\begin{cvitems}
\item This scholarship supports PhD students in the Faculty of Science undertaking travel to attend conferences, fieldwork, etc. as part of a Study Away request. Awarded \$2000.
\end{cvitems}}
    \cventry{Metascience Conference}{Metascience 2023 travel award}{}{2023}{\begin{cvitems}
\item \$300 USD travel award to attend the Metascience 2023 conference in Washington, D.C
\end{cvitems}}
    \cventry{Association for Interdisciplinary Metaresearch and Open Science (AIMOS)}{AIMOS top-up scholarship}{}{2022}{\begin{cvitems}
\item AIMOS will award up to four top-up scholarships per year to PhD or Masters students working on a meta-research project.
\end{cvitems}}
    \cventry{University of Melbourne}{Research Excellence Award for Interdisciplinary Research (Group Award)}{}{2022}{\begin{cvitems}
\item Nominees will have been collaborators in interdisciplinary research of outstanding influence, that is, the establishment of new, or advancing of existing, collaborations and programs that draw on multiple disciplines typically involving multiple faculties or schools.
\end{cvitems}}
    \cventry{Melbourne Centre for Data Science, University of Melbourne}{Melbourne Centre of Data Science Doctral Academy Fellow}{}{2021}{\begin{cvitems}
\item The MCDS Doctoral Academy aims to bring together a campus wide multi-disciplinary cohort of PhD students (MCDS Doctoral Academy Fellows) to share their research, domain challenges and thoughts around the use, implementation and application of data science in their fields.
\end{cvitems}}
    \cventry{The University of Melbourne}{Australian Government Research Training Program (RTP) Scholarship}{}{2017 - Current}{\begin{cvitems}
\item Awarded to high-achieving students undertaking graduate research at the University of Melbourne.
\end{cvitems}}
\end{cventries}

\hypertarget{publications}{%
\section{Publications}\label{publications}}

\begin{cventries}
    \cventry{Gould, E., Fraser, H., Parker, T.H. et al.}{Same data, different analysts: variation in effect sizes due to analytical decisions in ecology and evolutionary biology}{EcoEvoRxiv}{2023}{\begin{cvitems}
\item https://doi.org/10.32942/X2GG62
\end{cvitems}}
    \cventry{Ivimey-Cook, E., Pick, J.L., Bairos-Novak, K., Culina, A., Gould, E., Grainger, M., Marshall, B., Moreau, D., Paquet, M., Royauté, R., Sanchez-Tojar, A., Silva, I., Windecker, S.}{Implementing code review in the scientific workflow: Insights from ecology and evolutionary biology}{Journal of Evolutionary Biology}{2023}{\begin{cvitems}
\item https://doi.org/10.1111/jeb.14230
\end{cvitems}}
    \cventry{Fraser, H.,Bush, M., Wintle, B.C., Mody, F., Smith, E.T.,Hanea, A.M., Gould, E.,  Hemming, V., Hamilton, D.G., Rumpff, L., Wilkinson, D.P., Pearson, R., Singleton Thorn, F., Ashton, R., Willcox, A., Gray, C.T., Head, A., Ross, M., Groenewegen, R., Marcoci, A., Vercammen, A., Parker, T.H., Hoekstra, R., Nakagawa, S., Mandel, D.R., van Ravenzwaaij, D., McBride, M.F., Sinnott, R.O., Vesk, P.A., Burgman, M., Fidler, F.}{Predicting reliability through structured expert elicitation with the repliCATS (Collaborative Assessments for Trustworthy Science) process}{PLOS ONE}{2023}{\begin{cvitems}
\item https://doi.org/10.1371/journal.pone.0274429
\end{cvitems}}
    \cventry{Nakagawa, S., Ivimey-Cook, E., Grainger, M.J., O’Dea, R.E., Burke, S., Drobniak, S.M., Gould, E., Macartney, E.L., Martinig, A.R., Paquet, M., Morrison, K., Pick, J.L., Pottier, P., Ricolfi, L., Wilkinson, D.P., Willcox, A., Williams, C., Wilson, L.A.B., Windecker, S.M., Yang, Y., Lagisz, M.}{Method Reporting with Initials for Transparency (MeRIT) promotes more granularity and accountability for author contributions}{Nature Communications}{2023}{\begin{cvitems}
\item https://doi.org/10.1038/s41467-023-37039-1
\end{cvitems}}
    \cventry{Wintle, B.C., Mody, F., Smith, E.T.,Hanea, A.M., Wilkinson, D.P., Hemming, V., Bush, M., Fraser, H., Singleton Thorn, F., McBride, M.F., Gould, E., Head, A., Hamilton, D.G., Rumpff, L., Hoekstra, R., Fidler, F.}{Predicting and reasoning about replicability using structured groups}{Royal Society Open Science}{2023}{\begin{cvitems}
\item https://osf.io/preprints/metaarxiv/vtpmb/
\end{cvitems}}
    \cventry{Jones, C.S., Thomas, F.M., Michael, D.R., Fraser, H., Gould, E., Begley, J., Wilson, J., Vesk, P.A., Rumpff, L.}{What state of the world are we in? Targeted monitoring to detect transitions in vegetation restoration projects}{Ecological Applications}{2022}{\begin{cvitems}
\item https://doi.org/10.1002/eap.2728
\end{cvitems}}
    \cventry{Hanea, A.M., Wilkinson, D.P., McBride, M.F., Lyon, A., van Ravenzwaaij, D., Singleton Thorn, F., Gray, C.T., Mandel, D.R., Willcox, A., Gould, E., Smith, E.T., Mody, F., Bush, M., Fidler, F., Fraser, H., Wintle, B.C.}{Mathematically aggregating experts’ predictions of possible futures}{PLOS ONE}{2021}{\begin{cvitems}
\item https://doi.org/10.1371/journal.pone.0256919
\end{cvitems}}
    \cventry{O'Dea, R.E., Parker, T.H., Chee, Y.E., Culina, A., Drobniak, S.M., Duncan, D.H., Fidler, F., Gould, E., Ihle, M., Kelly, C., Lagisz, M., Roche, D.G., Sánchez-Tójar, A., Wilkinson, D.P., Wintle B.C., Nakagawa, S.}{Towards open, reliable, and transparent ecology and evolutionary biology}{BMC Biology}{2021}{\begin{cvitems}
\item https://doi.org/10.1186/s12915-021-01006-3
\end{cvitems}}
    \cventry{Good, M., Fraser, H., Gould, E., Vesk, P., Rumpff, L.}{A practical guide for conservation planning using the General Ecosystem Model for Southern Australian Woodlands.}{NESP Threatened Sprecies Recovery Hub Project 7.2, Brisbane}{2021}{\begin{cvitems}
\item https://www.nespthreatenedspecies.edu.au/media/yvlbs1tk/7-2-a-practical-guide-for-conservation-planning-using-the-general-ecosystem-model-for-southern-australian-woodlands\_v3.pdf
\end{cvitems}}
\end{cventries}

\hypertarget{selected-talks-and-workshops}{%
\section{Selected Talks and
Workshops}\label{selected-talks-and-workshops}}

\begin{cventries}
    \cventry{Many Analysts: Heterogeneity in results among studies in ecology and evolutionary biology}{Invited speaker Mini-note Panel: Association for Interdisciplinary Metaresearch and Open Science Conference}{Melbourne, Australia}{2022}{}\vspace{-4.0mm}
    \cventry{A many-analyst project in ecology and evolutionary biology demonstrates heterogeneity driven by analysts’ decisions and generates new questions about variability in this heterogeneity }{Big Team Science Conference}{Global - Online}{2022}{}\vspace{-4.0mm}
    \cventry{Workshop: Creating reproducible workflows in R with the targets:: package}{Society for Open Reliable and Transparent Ecology and Evoluation Workshop Series}{Online - Oceania}{2022}{}\vspace{-4.0mm}
    \cventry{'ResearchOps: A principled framework and guide to computational reproducibility', and 'Modelling as Ways of Knowing - How Viewing Models as an Epistemic Activity is useful in Ecology'}{Association for Interdisciplinary Metaresearch and Open Science Conference}{Melbourne, Australia}{2021}{}\vspace{-4.0mm}
    \cventry{Workshop: 'Preregistration Templates for Model-Based Research'}{Model Based Research and Reproducibility Workshop}{Centre for Open Science}{2020}{}\vspace{-4.0mm}
    \cventry{Poster Presentation: 'Questionable Research Practices in Non-Hypothesis Testing Research' }{MetaScience Symposium \& Association for Interdisciplinary Meta-Research and Open Science Conference}{San Francisco \& Melbourne}{2019}{}\vspace{-4.0mm}
\end{cventries}

\hypertarget{research-consultancy-professional-membership}{%
\section{Research Consultancy, Professional
Membership}\label{research-consultancy-professional-membership}}

\begin{cventries}
    \cventry{Preregistration Template Working Group}{Center of Open Science}{Professional Membership}{2023}{\begin{cvitems}
\item  the Preregistration Template Working Group is working to: 1. Establish criteria to evaluate the suitability of new preregistration templates for inclusion in the OSF. 2. Develop a procedure by which community creators of preregistration templates can put templates forward for inclusion in OSF., 3. Advise and inform COS on issues related to preregistration implementation in OSF.
\end{cvitems}}
    \cventry{Testing and Developing preregistration templates for ecology and conservation using a case study of environmental flows management in Victoria, Australia}{Victorian Government Department of Environment Land Water \& Planning}{Consultancy}{March 2020 - Present}{\begin{cvitems}
\item This research consultancy contributes to Elliot's PhD. This work involved the design and delivery of a collaborative workshop with DELWP in order to develop preregistration templates and methodology relevant to ecological modelling, in particular within decision-making and applied contexts.
\end{cvitems}}
    \cventry{Founding Member and Secretary / Treasurer, www.sortee.org}{Society for Open, Reliable, and Transparent Ecology and Evolutionary Biology (SORTEE)}{Professional Membership}{2020 - 2022, 2023}{\begin{cvitems}
\item SORTEE is a service organization which brings together researchers working to improve reliability and transparency through cultural and institutional changes in ecology, evolutionary biology, and related fields broadly defined.
\end{cvitems}}
\end{cventries}

\end{document}
